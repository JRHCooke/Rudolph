\documentclass{article}
\usepackage{graphicx}
\title{Arjan, software for an experiment with virtual 3D cubes and lateral chair}
\author{Wilbert van Ham}
\newcommand{\width}{\mathrm{width}}
\newcommand{\height}{\mathrm{height}}
\newcommand{\near}{\mathrm{near}}
\newcommand{\focal}{\mathrm{focal}}
\newcommand{\far}{\mathrm{far}}
\begin{document}
\maketitle
\section{Setup}
Image \ref{im:frustum} shows the frustum of the Sleelab setup. Note that compared to the standard 
openGL functions there is one extra variable, the distance to focal plane \textit{focal}.

\begin{figure}[hbt]
\includegraphics[height=7cm]{frustum.pdf}
\caption{The frustum}
\label{im:frustum}
\end{figure}
Equation \ref{eq:MVP} describes the modelview-projection matrix. The size of
the screen is $\mathrm{width}$ by $\mathrm{height}$. The $\mathrm{near}$,
$\mathrm{focal}$ and $\mathrm{far}$ distances are from the viewer to the near clipping
plane, focal plane (the screen) and the far clipping plane respectively. The viewer
is at coordinates $(x, y, z)$, looking in the negative z-direction with
the y-axis pointing up. 

The way the homogeneous matrix is represented makes it suitable for left multiplication 
with a row vector. For this matrix to be used in OpenGL, it must be stored
row-wise (C-style). The trace has the indices $0$, $4$ and $8$. The translations in the bottom 
row have the indices $12$, $13$ and $14$.

\begin{equation}
\mathrm{MVP}=\left(
\begin{array}{cccc}
 2\focal/\width &               0 &                                                     0 &     0 \\
              0 & 2\focal/\height &                                                     0 &     0 \\
     -2x/\width &      -2y/\height &                         \frac{\far+\near}{\near-\far} &    -1 \\
              0 &               0 & \frac{2\far\cdot\near-\focal(\far+\near)}{\near-\far} & \focal
\end{array} 
\right)
\label{eq:MVP}
\end{equation}
\section{Calculation}
This modelview-projection matrix is the multiplication of a gluLookAt from the observers position 
$(x_o, y_o, z_o)$ to the
origin with the y-axis pointing up (equation \ref{eq:lookat}), a glFrustum to a screen with size \textit{width}
$\times$ \textit{height} at distance \textit{focal} with clipping planes at distances \textit{near} and
\textit{far} (equation \ref{eq:frustum}), and a glTranslate over the normalized viewer position vector $(2x/\mathrm{width},
2y/\mathrm{height})$ (equation \ref{eq:translate}). Note the \textit{left} multiplication! Equating $z$ to \textit{focal}
causes the translation terms at index $12$ and $13$ to disappear. We used \textit{focal} for $z$ in the representation above.
\begin{equation}
\mathrm{gluLookAt}=\left(
\begin{array}{cccc}
  1 &  0 &  0 & 0 \\
  0 &  1 &  0 & 0 \\
  0 &  0 &  1 & 0 \\
 -x_o & -y_o & -z_o & 1
\end{array} 
\right)
\label{eq:lookat}
\end{equation}
\begin{equation}
\mathrm{glFrustum}=\left(
\begin{array}{cccc}
 2\focal/\width &               0 &                                  0 &     0 \\
              0 & 2\focal/\height &                                  0 &     0 \\
              0 &               0 &      \frac{\far+\near}{\near-\far} &    -1 \\
              0 &               0 & \frac{2\far\cdot\near}{\near-\far} &     0
\end{array} 
\right)
\label{eq:frustum}
\end{equation}
\begin{equation}
\mathrm{glTranslate}=\left(
\begin{array}{cccc}
 1 & 0 & 0 & 0 \\
 0 & 1 & 0 & 0 \\
 0 & 0 & 1 & 0 \\
 2x_o/\width & 2y_o/\height & 0 & 1
\end{array} 
\right)
\label{eq:translate}
\end{equation}

\subsection{Maxima}
Doing this with a symbolic algebra package is easier then by hand. We can use Maxima 
to see if this tranform ations does what we want.
\begin{verbatim}
/* xo = x observer , zo = z observer which is the distance to the focal plane f */

V: matrix([1,0,0,0],[0,1,0,0],[0,0,1,0],[-xo, -yo, -zo, 1]);
P: matrix([2*focal/width,0,0,0],[0,2*focal/height,0,0],[0,0,(far+near)/(near-far),-1],[0, 0, 2*far*near/(near-far), 0]);
B: matrix([1,0,0,0],[0,1,0,0],[0,0,1,0],[2*xo/width, 2*yo/height, 0, 1]);
MVP : V.P.B;

MVP : V.P.B, zo:focal;

/* convert from homogenuous to cartesian*/
cart(v) := matrix([v[1][1]/v[1][4],v[1][2]/v[1][4],v[1][3]/v[1][4]]);
/* convert from homogenuous to cartesian, ignore depth */
xy(v) := matrix([v[1][1]/v[1][4],v[1][2]/v[1][4]]);


xy(matrix([x,0,0,1]).MVP);
/* yields 2x/width, which is the correct position on the focal plane */
xy(matrix([x,y,z,1]).MVP);
/*yields
[ 2focal x 2xo z 2focal y 2yo z ]
[ --------- - ------  --------- - ------ ]
[   width   width  height   height]
[ ------------------  ------------------ ]
[     focal - z           focal - z      ]
which indeed describes a straight line from p_observer to the position on the focal plane.
*/
f(t) := (1-t)*matrix([xo,yo,focal,1]) + t*matrix([x,y,z,1]);
xy(f(t).MVP);

/* the correctness in the focal plane is easily verified:*/
xy(matrix([x,y,0,1]).MVP);
/* yields */
2x/width, 2y/height
\end{verbatim}
\end{document}
